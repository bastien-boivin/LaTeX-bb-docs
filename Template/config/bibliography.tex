%==============================================================================
% CONFIGURATION DE LA BIBLIOGRAPHIE
% Ce fichier configure la bibliographie pour le document.
% Il utilise le package biblatex pour gérer les références bibliographiques.
% Il est important de s'assurer que le fichier .bib est correctement configuré
% et que les références sont bien formatées.
%==============================================================================
\usepackage[
  backend=biber,
  style=authoryear-comp,
  sorting=nyt,
  maxcitenames=2,
  maxbibnames=99,
  giveninits=true,
  uniquename=init,
  doi=true,
  isbn=false,
  url=true,
  eprint=false,
  dashed=false,
  clearlang=true,
  abbreviate=true,
  autolang=hyphen,
]{biblatex}

\hypersetup{hypertexnames=true}

%------------------------------------------------------------------------------
% CRÉATION DES POINTS D'ANCRAGE POUR LES HYPERLIENS
%------------------------------------------------------------------------------
\makeatletter
\AtBeginDocument{%
  \AtEveryCitekey{%
    \hypertarget{cite.\thefield{entrykey}}{}%
  }
}

%------------------------------------------------------------------------------
% SUPPRESSION DES CROCHETS ET FORMATAGE CORRECT
%------------------------------------------------------------------------------
% Formatage correct pour l'année
\DeclareFieldFormat{labelyear}{#1}
\DeclareFieldFormat{extrayear}{}

% Délimiteurs pour apparence correcte
\DeclareDelimFormat{nameyeardelim}{\addcomma\space}
\DeclareDelimFormat{multicitedelim}{\addsemicolon\space}

%------------------------------------------------------------------------------
% REDÉFINITION COMPLÈTE DES MACROS DE CITATION
%------------------------------------------------------------------------------
% Citation parenthétique sans crochets
\renewbibmacro*{cite}{%
  \printtext[bibhyperref]{%
    \printnames{labelname}%
    \setunit{\nameyeardelim}%
    \printfield{year}%
  }%
}

% Citation textuelle sans crochets
\renewbibmacro*{textcite}{%
  \printtext[bibhyperref]{%
    \printnames{labelname}%
    \setunit{\addspace}%
    \printtext[parens]{\printfield{year}}%
  }%
}

% Configuration pour que bibhyperref utilise correctement le lien
\DeclareFieldFormat{bibhyperref}{%
  \bibhyperref{#1}%
}

\newrobustcmd{\bibhyperref}[1]{%
  \hyperlink{cite.\thefield{entrykey}}{#1}%
}

% Redéfinir les macros cite et textcite pour qu'ils fonctionnent avec des parenthèses
\DeclareCiteCommand{\parencite}[\mkbibparens]
  {\usebibmacro{prenote}}
  {\usebibmacro{citeindex}%
   \usebibmacro{cite}}
  {\multicitedelim}
  {\usebibmacro{postnote}}

\DeclareCiteCommand{\textcite}
  {\boolfalse{cbx:parens}}
  {\usebibmacro{citeindex}%
   \usebibmacro{textcite}}
  {\ifbool{cbx:parens}
     {\bibcloseparen\global\boolfalse{cbx:parens}}
     {}%
   \multicitedelim}
  {\usebibmacro{textcite:postnote}}

%------------------------------------------------------------------------------
% PERSONNALISATIONS SUPPLÉMENTAIRES
%------------------------------------------------------------------------------
\DeclareNameAlias{author}{family-given}
\DeclareFieldFormat{pages}{#1}
\DeclareFieldFormat{title}{\textit{#1}}

% Chargement du fichier .bib
\addbibresource{assets/bibliography/references.bib}

% Compatibilité avec les commandes standard
\let\citep\parencite
\providecommand{\citet}{\textcite}

%------------------------------------------------------------------------------
% TRADUCTIONS ET PERSONNALISATIONS
%------------------------------------------------------------------------------
\DefineBibliographyStrings{french}{%
  bibliography = {Références bibliographiques},
  references = {Références},
  page = {p\adddot},
  pages = {pp\adddot},
  in = {dans},
  volume = {vol\adddot},
  number = {n\textsuperscript{o}},
  editor = {éd\adddot},
  editors = {éds\adddot},
}

%------------------------------------------------------------------------------
% PARAMÈTRES D'AFFICHAGE DE LA BIBLIOGRAPHIE
%------------------------------------------------------------------------------
\setlength{\bibhang}{0.8cm}
\setlength{\bibitemsep}{0.5\baselineskip}

\defbibheading{bibliography}[\bibname]{%
  \chapter*{#1}%
  \markboth{#1}{#1}% 
}
\makeatother