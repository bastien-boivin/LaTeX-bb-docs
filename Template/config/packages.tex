%==============================================================================
% CONFIGURATION DES PACKAGES
%==============================================================================
% Ce fichier contient tous les packages nécessaires pour le template
%==============================================================================

%------------------------------------------------------------------------------
% ENCODAGE ET POLICES
%------------------------------------------------------------------------------
\usepackage[utf8]{inputenc}     % Prise en charge des caractères accentués
\usepackage[T1]{fontenc}        % Encodage des polices pour support du français
\usepackage[french]{babel}      % Support de la langue française
\usepackage{lmodern}            % Police Latin Modern
\usepackage{microtype}          % Améliorations typographiques fines
\usepackage{tgheros}            % Police sans-serif moderne (TeX Gyre Heros)

%------------------------------------------------------------------------------
% MISE EN PAGE ET GRAPHIQUES
%------------------------------------------------------------------------------
\usepackage{graphicx}           % Insertion d'images
\usepackage{tikz}               % Dessin vectoriel
\usetikzlibrary{calc}           % Extensions de calcul pour TikZ
\usepackage{float}              % Contrôle du positionnement (option [H])
\usepackage{fontawesome5}       % Icônes vectorielles

%------------------------------------------------------------------------------
% GESTION DES AVERTISSEMENTS
%------------------------------------------------------------------------------
\usepackage{silence}            % Pour désactiver certains avertissements
\WarningFilter{minitoc(hints)}{The titlesec package is loaded}
\WarningFilter{minitoc(hints)}{Some hints have been written}

%------------------------------------------------------------------------------
% TABLEAUX ET FIGURES
%------------------------------------------------------------------------------
\usepackage{booktabs,array}     % Tableaux professionnels
\usepackage{siunitx}            % Formatage des unités
% CORRECTION: Déplacé caption avant minitoc
\usepackage{caption}            % Personnalisation des légendes
\usepackage{subcaption}         % Pour les sous-figures

%------------------------------------------------------------------------------
% TABLES DES MATIÈRES
%------------------------------------------------------------------------------
\usepackage{minitoc}            % Tables des matières par chapitre
\setcounter{minitocdepth}{2}    % Profondeur (jusqu'aux sous-sections)
\setlength{\mtcindent}{0pt}     % Pas d'indentation 
\renewcommand{\mtcfont}{\small\rmfamily}       % Police mini-TOC
\renewcommand{\mtcSfont}{\small\rmfamily}      % Police titres mini-TOC

%------------------------------------------------------------------------------
% FORMATAGE ET STYLES
%------------------------------------------------------------------------------
\usepackage{titlesec}           % Personnalisation des titres
\usepackage{xcolor}             % Gestion des couleurs
\usepackage{fancyhdr}           % En-têtes et pieds de page personnalisés

%------------------------------------------------------------------------------
% CODE ET BOÎTES
%------------------------------------------------------------------------------
\usepackage{listings}           % Affichage de code source
\usepackage{tcolorbox}          % Boîtes colorées avancées
\tcbuselibrary{listings,skins,breakable}   % Extensions de tcolorbox

%------------------------------------------------------------------------------
% LIENS ET RÉFÉRENCES
%------------------------------------------------------------------------------
\usepackage[hidelinks,colorlinks=true,linkcolor=black,citecolor=black,urlcolor=blue]{hyperref}
\usepackage{bookmark}           % Améliorations des signets PDF
\usepackage{etoolbox}           % Macros utiles pour patcher des commandes