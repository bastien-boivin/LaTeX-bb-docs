%==============================================================================
% CONFIGURATION DE LA MISE EN PAGE
%==============================================================================
% Ce fichier contient les paramètres de mise en page du document
%==============================================================================

%------------------------------------------------------------------------------
% MARGES ET DIMENSIONS DE PAGE
%------------------------------------------------------------------------------
\usepackage[
  a4paper,           % Format A4
  top=2.5cm,         % Marge supérieure
  bottom=2.5cm,      % Marge inférieure
  left=2cm,          % Marge gauche
  right=2cm,         % Marge droite
  footskip=1cm,      % Espace pour le pied de page
  includehead=true,  % Inclure l'en-tête dans la hauteur de page
  includefoot=true,  % Inclure le pied de page dans la hauteur de page
  heightrounded      % Arrondir la hauteur à un nombre entier de lignes
]{geometry}

%------------------------------------------------------------------------------
% ESPACEMENTS
%------------------------------------------------------------------------------
% Espacement des paragraphes
\setlength{\parindent}{1em}      % Indentation des paragraphes
\setlength{\parskip}{0.5em plus 0.1em minus 0.1em}  % Espace entre paragraphes

% Espacement des listes
\usepackage{enumitem}            % Meilleur contrôle des listes
\setlist{nosep}                  % Supprime l'espace vertical supplémentaire
\setlist[itemize]{leftmargin=*}  % Alignement des puces avec le texte

%------------------------------------------------------------------------------
% RÉGLAGES DE JUSTIFICATION ET CÉSURE
%------------------------------------------------------------------------------
\usepackage{ragged2e}            % Contrôle amélioré de la justification
\tolerance=1000                  % Augmente la tolérance pour les lignes
\emergencystretch=10pt           % Aide LaTeX à éviter les dépassements
\hyphenpenalty=1000              % Diminue la probabilité de césure
\hbadness=10000                  % Supprime les avertissements de boîtes sous-remplies

%------------------------------------------------------------------------------
% RÉGLAGES POUR FIGURES ET TABLEAUX FLOTTANTS
%------------------------------------------------------------------------------
% Contrôle des positionnements des flottants
\renewcommand{\topfraction}{0.85}       % Max fraction de page pour flottant en haut
\renewcommand{\bottomfraction}{0.75}    % Max fraction de page pour flottant en bas
\renewcommand{\textfraction}{0.15}      % Min fraction de texte dans une page
\renewcommand{\floatpagefraction}{0.7}  % Min fraction pour une page de flottants

% Réglage des espaces entre flottants et texte
\setlength{\textfloatsep}{10pt plus 2pt minus 2pt}     % Espace entre texte et flottant
\setlength{\floatsep}{8pt plus 2pt minus 2pt}          % Espace entre flottants
\setlength{\intextsep}{8pt plus 2pt minus 2pt}         % Espace autour flottant dans texte

%------------------------------------------------------------------------------
% PERSONNALISATION DES LISTES
%------------------------------------------------------------------------------
% Configuration des listes numérotées
\setlist[enumerate,1]{label=\arabic*., ref=\arabic*}   % Premier niveau: 1., 2., ...
\setlist[enumerate,2]{label=\alph*), ref=\alph*}       % Second niveau: a), b), ...
\setlist[enumerate,3]{label=\roman*., ref=\roman*}     % Troisième niveau: i., ii., ...

% Configuration des listes à puces
\setlist[itemize,1]{label=\textbullet}                 % Premier niveau: •
\setlist[itemize,2]{label=\textendash}                 % Second niveau: –
\setlist[itemize,3]{label=\textasteriskcentered}       % Troisième niveau: *

%------------------------------------------------------------------------------
% GESTION DES LIGNES ORPHELINES ET VEUVES
%------------------------------------------------------------------------------
\widowpenalty=10000      % Évite fortement les veuves (dernière ligne d'un paragraphe seule en haut de page)
\clubpenalty=10000       % Évite fortement les orphelines (première ligne d'un paragraphe seule en bas de page)