% Document principal de suivi de thèse et documentation de code
\documentclass[12pt,a4paper]{report}

% ---------- PACKAGES DE BASE ----------
\usepackage[utf8]{inputenc}
\usepackage[T1]{fontenc}
\usepackage[french]{babel}
\usepackage{lmodern}
\usepackage{microtype}
\usepackage{graphicx}
\usepackage{tikz}
\usetikzlibrary{calc}

% ---------- MINI-TABLES DES MATIÈRES ----------
\usepackage{minitoc}
\setcounter{minitocdepth}{2}
\setlength{\mtcindent}{0pt}
\renewcommand{\mtcfont}{\small\rmfamily}
\renewcommand{\mtcSfont}{\small\rmfamily}

% ---------- MARGES ----------
\usepackage[a4paper,top=2.5cm,bottom=2.5cm,left=2cm,right=2cm]{geometry}

% ---------- STYLE DES CHAPITRES ----------
\usepackage{titlesec}
\titleformat{\chapter}[display]
  {\normalfont\bfseries\Huge\centering}
  {\rule{\textwidth}{1pt}\vspace{1cm}\Huge\chaptertitlename~\thechapter}
  {1cm}
  {\Huge}
  [\vspace{0.5cm}\rule{\textwidth}{1pt}]

\titlespacing*{\chapter}{0pt}{0pt}{50pt}

% ---------- EN-TÊTES ET PIEDS DE PAGE ----------
\usepackage{fancyhdr}
\pagestyle{fancy}
\fancyhf{}
\setlength{\headheight}{14.9pt}
\usepackage{xcolor}
\fancyhead[L]{\textcolor{gray}{Bastien Boivin}}
\fancyhead[R]{\textcolor{gray}{Suivi de Thèse}}
\fancyfoot[L]{\textcolor{gray}{Brouillon}}
\fancyfoot[R]{\textcolor{gray}{2025}}
\fancyfoot[C]{\thepage}
\renewcommand{\headrulewidth}{0.4pt}
\renewcommand{\footrulewidth}{0.4pt}
\renewcommand{\headrule}{\color{gray}\hrule height \headrulewidth}
\renewcommand{\footrule}{\color{gray}\hrule height \footrulewidth}

% ---------- STYLE POUR CODE PYTHON ----------
\usepackage{listings}
\usepackage[most]{tcolorbox}
\tcbuselibrary{listings,skins}
\definecolor{gdbg}{HTML}{0D1117}
\definecolor{gdfg}{HTML}{C9D1D9}
\definecolor{gdcomment}{HTML}{8B949E}
\definecolor{kdkeyword}{HTML}{FF7B72}
\definecolor{gdstring}{HTML}{A5D6FF}
\definecolor{gdidentifier}{HTML}{7DD7A6}
\definecolor{gdself}{HTML}{D2B48C}
\lstdefinestyle{githubdark}{%
  language=Python,
  backgroundcolor=\color{gdbg},
  basicstyle=\ttfamily\small\color{gdfg},
  keywordstyle=\color{kdkeyword}\bfseries,
  commentstyle=\color{gdcomment}\itshape,
  stringstyle=\color{gdstring},
  identifierstyle=\color{gdidentifier},
  emph={self}, emphstyle=\color{gdself}\bfseries,
  showstringspaces=false,
  numbers=left,
  numberstyle=\tiny\color{gdcomment},
  numbersep=6pt,
  breaklines=true,
  frame=none,
  captionpos=b,
  tabsize=4,
  inputencoding=utf8,
  extendedchars=true,
  literate={é}{{\'e}}1 {è}{{\`e}}1 {ê}{{\^e}}1 {à}{{\`a}}1 {ç}{{\c{c}}}1
}
\tcbset{
  pythonlisting/.style={
    enhanced, breakable, sharp corners=downhill,
    boxrule=0pt, colback=gdbg, colframe=gdbg,
    left=3pt, right=3pt, top=3pt, bottom=3pt,
    listing only,
    listing options={style=githubdark},
  }
}
\newtcblisting{pythoncode}[1][]{pythonlisting, title=#1}

% ---------- DÉBUT DU DOCUMENT ----------
\begin{document}

\dominitoc % Activer les mini-tables des matières

% --- Page de garde ---
% sections/page_de_garde.tex
\begin{titlepage}
    \thispagestyle{empty}
    \begin{tikzpicture}[remember picture, overlay]
      \draw[line width=1pt]
        ($(current page.north west)+(1cm,-1cm)$) rectangle
        ($(current page.south east)+(-1cm,1cm)$);
    \end{tikzpicture}
    \begin{center}
      \includegraphics[height=1.5cm]{figures/logos/Logo_Univ_Rennes.png}\hspace{1cm}%
      \includegraphics[height=1.5cm]{figures/logos/Logo_EBR.png}\hspace{1cm}%
      \includegraphics[height=1.5cm]{figures/logos/Logo_Fondation_Rennes.png}
      \vspace{2cm}
  
      {\Huge\textbf{Suivi de Thèse en Modélisation Hydrogéologique}}\\[0.5cm]
      {\Large Documentation de Code et Avancées Scientifiques}
      \vspace{1.5cm}
  
      {\Large\textcolor{red}{\textbf{Version Brouillon – Document de Travail}}}
      \vspace{2cm}
  
      {\large
        \textbf{Auteur :} Bastien Boivin\\[0.3cm]
        \textbf{Email (pro) :} bastien.boivin@univ-rennes.fr\\
        \textbf{Email (perso) :} bastien.boivin@proton.me\\[1cm]
      }
  
      \noindent
      \begin{flushleft}
        \textbf{Directeur de thèse :}\\
        Jean-Raynald de Dreuzy, Directeur de recherche CNRS, Géosciences Rennes\\[0.5cm]
        \textbf{Co-directeur de thèse :}\\
        Luc Aquilina, Professeur des universités, Géosciences Rennes\\[0.5cm]
        \textbf{Partenaire industriel :}\\
        Jean-Yves Gaubert, Directeur du pôle R\&D, Eau du Bassin Rennais\\
      \end{flushleft}
  
      \vfill
      Rennes, \today
    \end{center}
  \end{titlepage}
  

% --- Table des matières générale ---
\tableofcontents

% --- Style fancyhdr pour la suite ---
\pagestyle{fancy}

% --- Contenu principal ---
%==============================================================================
% CHAPITRE D'INTRODUCTION
%==============================================================================

\chapter{Introduction}

% Mini-table des matières du chapitre
\minitoc

\newpage

%------------------------------------------------------------------------------
% 1
%------------------------------------------------------------------------------


\end{document}