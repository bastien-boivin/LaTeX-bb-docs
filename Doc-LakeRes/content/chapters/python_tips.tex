%==============================================================================
% CHAPITRE PYTHON TIPQ
%==============================================================================

\chapter{Python Tips}
\label{chap:python_tips}

% Mini-table des matieres du chapitre
\minitableofcontents

\newpage

%------------------------------------------------------------------------------
% CONTENU
%------------------------------------------------------------------------------

\section{Decorators}
\label{sec:decorators}

\subsection{@StaticMethod}
\label{sec:staticmethod}

\inlinecode{@staticmethod} est un décorateur qui transforme une méthode de classe en une méthode statique. Cela signifie que la méthode peut être appelée sur la classe elle-même, sans avoir besoin d'une instance de la classe. Les méthodes statiques n'ont pas accès à l'instance (\inlinecode{self}) ou à la classe (\inlinecode{cls}) et ne peuvent pas modifier l'état de l'objet ou de la classe.

\begin{TipBox}
    Elle fonctionne comme une fonction normale, mais organisée dans l'espace de noms de la classe.
\end{TipBox}

\vspace{1em}


\noindent\textbf{Exemple :}

\begin{minted}{python}
class MathUtils:
    @staticmethod
    def add(x, y):
        return x + y
\end{minted}

\begin{ImportantBox}
    \begin{itemize}
        \item Ne pas tenter d'accéder à \inlinecode{self} dans une méthode statique.
        \item Ne pas utiliser \inlinecode{@staticmethod} quand la fonction à besoin d'accéder aux données de l'objet.
        \item Ne pas confondre avec des fonctions globales - les méthodes statiques ont un lien conceptuel avec leur classe.
    \end{itemize}
\end{ImportantBox}