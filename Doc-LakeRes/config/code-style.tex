%==============================================================================
% CONFIGURATION POUR L'AFFICHAGE DU CODE
%==============================================================================
% Ce fichier contient la configuration pour l'affichage du code source
% Il configure les styles visuels et les environnements de code
%==============================================================================

%------------------------------------------------------------------------------
% CONFIGURATION DU STYLE DE CODE PYTHON (GITHUB DARK)
%------------------------------------------------------------------------------
% Note: Les couleurs sont définies dans colors.tex

\lstdefinestyle{githubdark}{%
  language=Python,
  backgroundcolor=\color{gdbg},
  basicstyle=\ttfamily\small\color{gdfg},
  keywordstyle=\color{kdkeyword}\bfseries,
  commentstyle=\color{gdcomment}\itshape,
  stringstyle=\color{gdstring},
  identifierstyle=\color{gdidentifier},
  emph={self},emphstyle=\color{gdself}\bfseries,
  showstringspaces=false,
  numbers=left,numberstyle=\tiny\color{gdcomment},numbersep=12pt,
  breaklines=true,frame=none,captionpos=b,tabsize=4,
  inputencoding=utf8,extendedchars=true,
  literate={é}{{\'e}}1 {è}{{\`e}}1 {ê}{{\^e}}1 {à}{{\`a}}1 {ç}{{\c{c}}}1
}

%------------------------------------------------------------------------------
% ENVIRONNEMENT POUR BLOC DE CODE PYTHON
%------------------------------------------------------------------------------
\tcbset{
  pythonlisting/.style={enhanced,breakable,arc=4pt,boxrule=0pt,
    colback=gdbg,colframe=gdbg,left=4pt,right=4pt,top=4pt,bottom=4pt,
    listing only,listing options={style=githubdark},
  }
}

\newtcblisting{pythoncode}[1][]{pythonlisting,title=#1}

%------------------------------------------------------------------------------
% CODE EN LIGNE
%------------------------------------------------------------------------------
% Style pour le code en ligne
\lstdefinestyle{inlinestyle}{
  basicstyle=\ttfamily\small,
  backgroundcolor=\color{inlineCodeBg},
  breaklines=true,
  keepspaces=true,
  columns=flexible,
  showstringspaces=false,
  showspaces=false
}

% Configuration globale de lstinline
\lstset{style=inlinestyle}

% Commande pour code en ligne avec formatage cohérent
\newcommand{\inlinecode}[1]{%
  \colorbox{inlineCodeBg}{%
    \lstinline[basicstyle=\ttfamily\small\color{inlineCodeText}]!#1!%
  }%
}

%------------------------------------------------------------------------------
% CHEMINS DE FICHIERS STYLISÉS
%------------------------------------------------------------------------------

% Chemin de fichier simple
\newcommand{\filepath}[1]{%
  \colorbox{pathBg}{%
    \lstinline[basicstyle=\ttfamily\small\color{pathText}]!#1!%
  }%
}

% Chemin avec icône de dossier
\newcommand{\folderpath}[1]{%
  \colorbox{pathBg}{%
    \textcolor{folderColor}{\faFolder}\,%
    \lstinline[basicstyle=\ttfamily\small\color{pathText}]!#1!%
  }%
}

% Chemin URL
\newcommand{\urlpath}[1]{%
  \colorbox{urlBg}{%
    \textcolor{linkColor}{\faLink}\,%
    \lstinline[basicstyle=\ttfamily\small\color{urlText}]!#1!%
  }%
}