%==============================================================================
% CONFIGURATION DES FIGURES ET TABLEAUX
%==============================================================================
% Ce fichier définit le style et les paramètres des figures et tableaux
%==============================================================================

%------------------------------------------------------------------------------
% CONFIGURATION DES TABLEAUX
%------------------------------------------------------------------------------
% Packages pour tableaux améliorés
\usepackage{booktabs}           % Règles horizontales de qualité typographique
\usepackage{array}              % Fonctionnalités étendues pour les tableaux
\usepackage{siunitx}            % Formatage cohérent des nombres et unités

% Espacement des tableaux
\renewcommand{\arraystretch}{1.2}   % Espacement vertical des lignes
\setlength{\tabcolsep}{8pt}         % Espacement horizontal des colonnes

% Configuration pour les nombres dans les tableaux
\sisetup{
  reset-text-family = false,     % CORRECTION: Remplacé detect-family
  text-family-to-math = true,    % CORRECTION: Ajouté pour remplacer detect-family
  table-number-alignment=center,  % Alignement central des nombres
  table-format=3.2,               % CORRECTION: Remplacé table-figures-*
  output-decimal-marker=.,        % Utilise le point comme séparateur décimal
  exponent-mode=fixed             % Format fixe pour les exposants
}

%------------------------------------------------------------------------------
% CONFIGURATION DES LÉGENDES
%------------------------------------------------------------------------------
% Définition des polices pour les légendes
\DeclareCaptionFont{figcaption}{\small\sffamily}       % Police du texte de légende
\DeclareCaptionFont{figlabel}{\bfseries\sffamily}      % Police de l'étiquette (Figure X:)

% Style pour légendes de figures
\captionsetup[figure]{
    format=hang,                % Format suspendu (étiquette en dehors de la marge)
    font=figcaption,            % Police du texte
    labelfont=figlabel,         % Police de l'étiquette
    labelsep=quad,              % Espace après l'étiquette
    justification=raggedright,  % Alignement à gauche
    singlelinecheck=true,       % Centrage automatique des légendes courtes
    skip=10pt,                  % Espace après la figure, avant la légende
    hypcap=false                % CORRECTION: Désactivé hypcap par défaut
}

% Style pour légendes de tableaux
\captionsetup[table]{
    format=hang,                % Format suspendu
    font=figcaption,            % Police du texte
    labelfont=figlabel,         % Police de l'étiquette
    labelsep=quad,              % Espace après l'étiquette
    justification=centering,    % Alignement centré pour les tableaux
    singlelinecheck=true,       % Centrage automatique des légendes courtes
    skip=5pt,                   % Espace après le tableau, avant la légende
    hypcap=false                % CORRECTION: Désactivé hypcap par défaut
}

% Style pour sous-figures
\captionsetup[subfigure]{
    format=hang,                % Format suspendu
    font=figcaption,            % Police du texte
    labelfont=figlabel,         % Police de l'étiquette
    labelsep=space,             % Espace simple après l'étiquette
    justification=centering,    % Alignement centré
    singlelinecheck=true,       % Centrage automatique des légendes courtes
    skip=5pt                    % Espace après la sous-figure
}

% Traduction des titres des listes de figures et tableaux
\renewcommand{\listfigurename}{Liste des figures}
\renewcommand{\listtablename}{Liste des tableaux}

%------------------------------------------------------------------------------
% ENVIRONNEMENT POUR FIGURES CENTRÉES
%------------------------------------------------------------------------------
% Commande facilitant l'insertion de figures
\newcommand{\centeredfigure}[3][1.0]{%
    \begin{center}
        \includegraphics[width=#1\textwidth]{#2}
        \captionof{figure}{#3}
        \label{fig:#2}
    \end{center}
}

%------------------------------------------------------------------------------
% COMMANDES POUR SOUS-FIGURES
%------------------------------------------------------------------------------
% Facilite la création de groupes de sous-figures (2 côte à côte)
\newcommand{\twofigures}[6][0.45]{%
    % #1 = largeur relative [optionnel, défaut=0.45]
    % #2, #4 = chemins des images
    % #3, #5 = légendes des sous-figures
    % #6 = légende principale
    \begin{figure}[htbp]
        \centering
        \begin{subfigure}[b]{#1\textwidth}
            \centering
            \includegraphics[width=\textwidth]{#2}
            \caption{#3}
            \label{fig:#2}
        \end{subfigure}
        \hfill
        \begin{subfigure}[b]{#1\textwidth}
            \centering
            \includegraphics[width=\textwidth]{#4}
            \caption{#5}
            \label{fig:#4}
        \end{subfigure}
        \caption{#6}
        \label{fig:#2-#4}
    \end{figure}
}

% Facilite la création de groupes de sous-figures (3 côte à côte)
\newcommand{\threefigures}[8][0.3]{%
    % #1 = largeur relative [optionnel, défaut=0.3]
    % #2, #4, #6 = chemins des images
    % #3, #5, #7 = légendes des sous-figures
    % #8 = légende principale
    \begin{figure}[htbp]
        \centering
        \begin{subfigure}[b]{#1\textwidth}
            \centering
            \includegraphics[width=\textwidth]{#2}
            \caption{#3}
            \label{fig:#2}
        \end{subfigure}
        \hfill
        \begin{subfigure}[b]{#1\textwidth}
            \centering
            \includegraphics[width=\textwidth]{#4}
            \caption{#5}
            \label{fig:#4}
        \end{subfigure}
        \hfill
        \begin{subfigure}[b]{#1\textwidth}
            \centering
            \includegraphics[width=\textwidth]{#6}
            \caption{#7}
            \label{fig:#6}
        \end{subfigure}
        \caption{#8}
        \label{fig:#2-#4-#6}
    \end{figure}
}

%------------------------------------------------------------------------------
% TABLEAUX PERSONNALISÉS
%------------------------------------------------------------------------------
% Environnement pour tableaux avec en-têtes en gras
\newenvironment{nicetable}[3]{%
    % #1 = légende
    % #2 = label
    % #3 = colonnes (ex: {lcc})
    \begin{table}[htbp]
    \centering
    \caption{#1}
    \label{#2}
    \begin{tabular}{#3}
    \toprule
}{%
    \bottomrule
    \end{tabular}
    \end{table}
}

% Commande pour ligne d'en-tête en gras
\newcommand{\tableheader}[1]{%
    \textbf{#1}%
}