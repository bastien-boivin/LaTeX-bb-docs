%==============================================================================
% COMMANDES PERSONNALISÉES UTILES
%==============================================================================
% Ce fichier contient des commandes LaTeX personnalisées pour faciliter
% la rédaction et améliorer la cohérence du document
%==============================================================================

%------------------------------------------------------------------------------
% RACCOURCIS POUR LES RÉFÉRENCES
%------------------------------------------------------------------------------
% Référence à une figure avec préfixe
\newcommand{\figref}[1]{Figure~\ref{fig:#1}}

% Référence à un tableau avec préfixe
\newcommand{\tabref}[1]{Tableau~\ref{tab:#1}}

% Référence à une équation avec préfixe
\newcommand{\eqref}[1]{Équation~(\ref{eq:#1})}

% Référence à une section avec préfixe
\newcommand{\secref}[1]{Section~\ref{sec:#1}}

%------------------------------------------------------------------------------
% MISE EN ÉVIDENCE ET FORMATAGE
%------------------------------------------------------------------------------
% Surlignage de texte avec couleur personnalisable
\newcommand{\highlight}[2][yellow]{\colorbox{#1}{#2}}

% Texte "TODO" en rouge pour les notes pendant la rédaction
\newcommand{\todo}[1]{\textcolor{red}{\textbf{TODO:} #1}}

% Annotation de relecture en marge
\newcommand{\note}[1]{\marginpar{\footnotesize\textcolor{gray}{#1}}}

%------------------------------------------------------------------------------
% UNITÉS ET NOTATIONS SCIENTIFIQUES
%------------------------------------------------------------------------------
% Unités cohérentes avec siunitx (permet également la mise à jour globale)
\newcommand{\ms}{\si{\meter\per\second}}
\newcommand{\cms}{\si{\centi\meter\per\second}}
\newcommand{\m}{\si{\meter}}
\newcommand{\cm}{\si{\centi\meter}}
\newcommand{\mm}{\si{\milli\meter}}
\newcommand{\km}{\si{\kilo\meter}}
\newcommand{\sqkm}{\si{\square\kilo\meter}}
\newcommand{\cubicm}{\si{\cubic\meter}}

% Notation mathématique spécifique au domaine
\newcommand{\porosity}{\ensuremath{\phi}}
\newcommand{\permeability}{\ensuremath{K}}
\newcommand{\conductivity}{\ensuremath{\sigma}}

%------------------------------------------------------------------------------
% ENVIRONNEMENTS ET BOÎTES PERSONNALISÉS
%------------------------------------------------------------------------------
% Environnement pour équation avec étiquette et référence automatique
\newenvironment{myeqnarray}[1]
  {\begin{equation}\label{eq:#1}}
  {\end{equation}}

% Bloc de texte encadré simple
\newcommand{\textframebox}[1]{%
  \begin{center}
    \fbox{\begin{minipage}{0.9\textwidth}#1\end{minipage}}
  \end{center}
}

%------------------------------------------------------------------------------
% COMMANDES POUR LES TITRES ET EN-TÊTES
%------------------------------------------------------------------------------
% Informations sur le document à définir au début pour utilisation cohérente
\newcommand{\documenttitle}{Titre du document}
\newcommand{\documentauthor}{Nom de l'auteur}
\newcommand{\documentdate}{\today}
\newcommand{\documentversion}{Version 1.0}

% Pour mettre à jour les en-têtes avec ces informations:
\newcommand{\updateheaders}{%
  \fancyhead[L]{\textcolor{gray}{\documentauthor}}
  \fancyhead[R]{\textcolor{gray}{\documenttitle}}
  \fancyfoot[L]{\textcolor{gray}{\documentversion}}
  \fancyfoot[R]{\textcolor{gray}{\documentdate}}
}

%------------------------------------------------------------------------------
% COMMANDES POUR GESTION DES CHEMINS ET FICHIERS
%------------------------------------------------------------------------------
% Les commandes filepath, folderpath et urlpath sont définies dans code-style.tex

% Chemin vers un script Python
\newcommand{\pyfile}[1]{\filepath{#1.py}}

% Chemin vers un fichier de données
\newcommand{\datafile}[1]{\filepath{data/#1}}

% Chemin vers un fichier de configuration
\newcommand{\configfile}[1]{\filepath{config/#1}}

